\documentclass[12pt]{article}
\usepackage[top=1in,bottom=1in,left=1in,right=1in]{geometry}
\usepackage{alltt}
\usepackage{array}	
\usepackage{graphicx}
\usepackage{tabularx}
\usepackage{verbatim}
\usepackage{setspace}
\usepackage{listings}
\usepackage{amssymb,amsmath, amsthm}
\usepackage{qtree}
\usepackage{hyperref}
\usepackage{oz}
\usepackage[cc]{titlepic}
\usepackage{fancyvrb}
\usepackage{epstopdf}

\title{SOEN 331 (Section): Introduction to Formal Methods\\for Software Engineering\\
\ \\
Assignment 3 on Temporal Logic}
\author{\textbf{Name(s)}\\
		\texttt{Email(s)}
\ \\}
\date{Date}

\begin{spacing}{1.5}
\begin{document}
\maketitle

\newpage

\section*{Problem 1 (20 pts):  Analyzing program behavior}

\begin{enumerate}

\item (10 pts) Visualize all models of behavior.

%
%\begin{figure}[h!]
%  \centering
%  \includegraphics[width=0.9\textwidth]{file.eps}
%\end{figure}

\item (3 pts) Specify conditions (model of behavior), if any exist, under which the program can terminate.

\item (7 pts) For the expressions below, indicate (true/false) whether there exists a 
model where the expression holds. When true, cross reference your particular model:

\begin{table}
\centering
\begin{tabular}{|l|l|}
\hline
\textbf{PROPERTY}							& \textbf{TRUE/FALSE}\\
\hline

$(a \wedge c) \rightarrow \Diamond \Box (g \wedge h)$	 &\\

&\\

\hline

&\\
		
$h ~\mathcal{U}~ m$									 &\\

&\\

\hline

&\\
		
$h ~\mathcal{U}~ (k \wedge g)$						 &\\

&\\

\hline

&\\
		
$(b \wedge c) \rightarrow \Box \Diamond (b \wedge c)$  &\\

&\\

\hline

&\\
		
$(k \wedge \bigcirc (k \wedge g)) \rightarrow \bigcirc m$  &\\

&\\

\hline

&\\
		
$ h ~\mathcal{S}~ c$								 &\\

&\\

\hline

&\\
		
$ ((g \wedge h) \wedge \bigcirc d) \rightarrow \bigcirc^{2} (g \wedge h)$  &\\

&\\

\hline

&\\
		
$e ~\mathcal{R}~ h$									 &\\

&\\

\hline

&\\
		
The program has the following stability property: &\\
$\Diamond \Box (b \wedge \ c \wedge h)$		 &\\

&\\

\hline

&\\
		
The program has the following response property: &\\
$\Box \Diamond (b \wedge \ c \wedge h)$		 &\\

&\\

\hline

&\\
		
$( g \wedge h)$ is an invariant property of the program.  &\\

&\\

\hline

&\\
		
There is a guarantee that $(g \wedge k \wedge h)$	 &\\

&\\

\hline

&\\
		
The program has the following response property: &\\
$(b \wedge c \wedge h) \rightarrow \Diamond (b \wedge c \wedge h)$.   &\\

&\\

\hline

&\\
		
The program has the following precedence property: &\\
$(b \wedge c \wedge h) \rightarrow ( (g \wedge h) ~\mathcal{U}~ b \wedge c \wedge h))$
			 &\\
&\\

\hline

\end{tabular}
\end{table}


\end{enumerate}


\newpage

\section*{Problem 2 (20 pts) :  Visualizing temporal expressions}

\begin{enumerate}

	\item $\Box (\phi \rightarrow \bigcirc^{2} \psi)$

	\item $\Box \phi \rightarrow \bigcirc \psi$
	
	\item $\phi \rightarrow \bigcirc \Diamond \Box \psi$
	
	\item $(\phi \wedge \bigcirc \psi) \rightarrow \bigcirc^{2} \Diamond \Box \omega$
	
	\item $\Box ((\phi \wedge \bigcirc \psi) \rightarrow \bigcirc^{2} \Diamond \Box \omega)$
	
	\item $(\phi \wedge \bigcirc \psi) \rightarrow \tau ~\mathcal{R}~ \upsilon$
	
	\item $(\phi \wedge \bigcirc \psi) \rightarrow \bigcirc (\tau ~\mathcal{R}~ \upsilon)$

	\item $(\phi \wedge \bigcirc \psi) \rightarrow \bigcirc (x ~\mathcal{U}~ \tau)$
	
	\item $(\phi \wedge \Box \psi) \rightarrow \bigcirc^{2} \Diamond \omega$
	
	\item $(\phi \wedge \bigcirc^{2} \psi) \rightarrow \bigcirc \Box \omega$

\end{enumerate}

%\begin{figure}[h!]
%  \centering
%  \includegraphics[width=0.8\textwidth]{file.eps}
%\end{figure}



\end{spacing}
\end{document}